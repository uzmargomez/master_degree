\documentclass{article}
\usepackage[letterpaper,margin=1in]{geometry}
\usepackage[dvipsnames]{xcolor}
\usepackage{fancyhdr}
\usepackage{tgschola} % or any other font package you like
\usepackage{titlesec}
\usepackage{comment}
\usepackage{amsmath}
\usepackage{setspace}
\usepackage[square,sort,comma,numbers]{natbib}
\usepackage{hyperref}
\hypersetup{
    colorlinks=true,
    linkcolor=blue,
    filecolor=magenta,      
    urlcolor=NavyBlue,
}


\pagestyle{fancy}
\fancyhf{}
\fancyhead[C]{%
  \footnotesize\sffamily
  \yourname:\quad
  {\youremail}}

\newcommand{\soptitle}{Statement of Purpose}
\newcommand{\yourname}{Uzmar G\'omez}
\newcommand{\youremail}{uzmar.gomez@ciencias.unam.mx}
\newcommand{\yourweb}{https://uzmargomez.github.io/}

\newcommand{\statement}[1]{\par\medskip
  \underline{{\textbf{#1:}}}\space
}

\titleformat{\section}{\large\itshape}{\thesection}{1em}{}

\renewcommand{\bibsection}{\section*{Bibliography}}

\begin{document}
\spacing{1.5}
\begin{center}\LARGE\soptitle\\
\large of \yourname\ (Durham University, MSc in Scientific Computing and Data Analysis)
\end{center}

\hrule
\vspace{1pt}
\hrule height 1pt

\bigskip

Physics has experienced an increasing interest over the last years, particularly in the field of astrophysics. We can find examples of this back in 2016 when the first gravitational wave signal was detected, or the continuous studies regarding the dark matter. Similarly, computer science has evolved rapidly, especially in the field of machine learning and cloud computing, allowing scientists to study big amounts of data from experiments and numerical simulations, and extract essential information out of them. As the amount of data from astrophysical experiments increase, it is natural for physicists to became self-proficient in the latest computer science tools and techniques.

I discovered my interest in both fields when I began my undergraduate program at the National Autonomous University of Mexico (\href{https://www.unam.mx/}{UNAM}). During my time in the Physics career, I started studying Relativity and Numerical Methods. Since then, I have been fascinated by the different methods humanity has used to understand bodies that are so far away, namely Black Holes, Neutron Stars or White Dwarfs. The fact that we cannot even see some of these bodies, but still know of their existence because of the influence they have on their surroundings, or because of the detection of some signal that agrees with our simulations, is just astonishing. Computers are an essential part of these studies since they are the only ones with the ability to solve the complex equations that model the behaviour of these bodies, as well as the only ones capable of analysing the enormous amount of information collected by our experiments. Because of all this, I was naturally attracted to the field of Numerical Relativity.

During my career, I started a job as a Professor Assistant of courses such as Mathematics and Relativity on the Faculty of Science UNAM, and I worked on a \href{http://oreon.dgbiblio.unam.mx/F/YF3LAD8UIUFR2XRPDE3H6JM2MI44M8RY8QX1RH7K429163KKEV-19694?func=find-b&request=uzmar&find_code=WRD&adjacent=N&local_base=TES01&x=65&y=13&filter_code_2=WYR&filter_request_2=&filter_code_3=WYR&filter_request_3=}{thesis} under the supervision of  \href{https://sigi.nucleares.unam.mx/sgiicn/people/user/view/id/8}{Dr Miguel Alcubierre Moya}, one of the leading researchers on the field of Numerical Relativity in Mexico. This thesis was mainly concerned with the numerical simulation of the evolution of the Vlasov Equation with a background gravitational field represented by the Schwarzschild metric. While working on this thesis, I learned more about Finite Difference Methods and the Fortran Programming Language. 

I have a passion for research, but also a great interest in improving my professional skills. This led me to read about other computer science topics, such as Machine Learning and Neural Networks, among others, while improving my programming ability. This knowledge allowed me to get a job as a Data Scientist, where I have worked in projects such as Face Detection and Recognition (using Neural Networks and Sparse Representation of matrices), the creation of a Chatbot using the Rasa framework, recommendation systems using the apriori algorithm, as well as churn prediction models using SVM and XGBoost. I also got hands-on experience with developing production-level code in different object-oriented programming languages (Python and C++), Google Cloud Platform services, containers and orchestration, machine and deep learning models, parallel computing using CUDA, and a basic understanding of algorithms and data structures. Considering this, I decided to ask for references from my current and former employers. I know it is preferred to have two academic references but, to be honest, I never really worked with anyone besides Dr Alcubierre almost three years ago. Having said that, I strongly believe I have become a better scientist and an overall better worker and student because of my time in the industry. I have learned much more, and I know that my employers have a great opinion about me, my abilities, and my possibilities to succeed in this program. If you would like to take a closer look at some of the projects I have work on, please visit this \href{https://github.com/uzmargomez/master_degree}{github repository}. 

My main goal in life is to contribute to the worldwide physics knowledge, using the modern tools and technologies we have available, and the new algorithms and techniques that are been applied right now in different areas and whose usage has not yet been completely exploited in Physics. To achieve this, I plan to implement an open-source tool to be used for experimenting and make astrophysics simulations, but that can also be generalized to other branches of science.

I believe that the MSc in Scientific Computing and Data Analysis is the perfect program for me, as it aims to teach how to improve the study of Physics using Computer Science, and I truly believe this is the future of science. I found that this program is unique in several different ways, emphasizing the great variety and flexibility of its courses, and the possibility to focus on one specific area between Astrophysics, Particle Physics, Earth and Environmental Sciences and Financial Mathematics. I expect this will allow me to meet people with goals similar to mine, but with different perspectives of what Computer Science can do for each one of the distinct disciplines. In addition to this, your postgraduate program stands out as one of the best and most unique in the world in terms of innovation and partnerships with important companies, and that is why I am so interested in becoming one of your students.

My commitment is to take advantage of the opportunity to be part of your program and dedicate myself to learn as much as I can, so that I can become a better person and scientist and contribute as much as I can to the world’s scientific and technological development in the upcoming future.

\vspace{4cm}
\hfill \textbf{Uzmar Gómez}

\hfill BSc Physics (UNAM), 

\hfill Data Scientist I at Rackspace Technology
\end{document}