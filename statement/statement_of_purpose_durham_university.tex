\documentclass{article}
\usepackage[letterpaper,margin=1in]{geometry}
\usepackage[dvipsnames]{xcolor}
\usepackage{fancyhdr}
\usepackage{tgschola} % or any other font package you like
\usepackage{titlesec}
\usepackage{comment}
\usepackage{amsmath}
\usepackage{setspace}
\usepackage[square,sort,comma,numbers]{natbib}
\usepackage{hyperref}
\hypersetup{
    colorlinks=true,
    linkcolor=blue,
    filecolor=magenta,      
    urlcolor=NavyBlue,
}


\pagestyle{fancy}
\fancyhf{}
\fancyhead[C]{%
  \footnotesize\sffamily
  \yourname:\quad
  {\youremail}}

\newcommand{\soptitle}{Statement of Purpose}
\newcommand{\yourname}{Uzmar G\'omez}
\newcommand{\youremail}{uzmar.gomez@ciencias.unam.mx}
\newcommand{\yourweb}{https://uzmargomez.github.io/}

\newcommand{\statement}[1]{\par\medskip
  \underline{{\textbf{#1:}}}\space
}

\titleformat{\section}{\large\itshape}{\thesection}{1em}{}

\renewcommand{\bibsection}{\section*{Bibliography}}

\begin{document}
\spacing{1.5}
\begin{center}\LARGE\soptitle\\
\large of \yourname\ (Durham University, MSc in Scientific Computing and Data Analysis)
\end{center}

\hrule
\vspace{1pt}
\hrule height 1pt

\bigskip

Here I expose the reasons I have to get into the MSc in Scientific Computing and Data Analysis.

As you can see in my curriculum,  I have a BSc in Physics. To get my Physics degree, I investigated about finite difference methods to solve advective-kind equations. In particular, I needed to evolve an equation called the Relativistic Vlasov Equation on a background Schwarzschild metric. I made a \href{http://oreon.dgbiblio.unam.mx/F/YF3LAD8UIUFR2XRPDE3H6JM2MI44M8RY8QX1RH7K429163KKEV-19694?func=find-b&request=uzmar&find_code=WRD&adjacent=N&local_base=TES01&x=65&y=13&filter_code_2=WYR&filter_request_2=&filter_code_3=WYR&filter_request_3=}{thesis} about this work, under \href{https://sigi.nucleares.unam.mx/sgiicn/people/user/view/id/8}{Dr. Miguel Alcubierre}'s guidance, at the \href{https://www.unam.mx/}{National Autonomous University of Mexico (UNAM)}. 

I've been a teacher assistant of subjects such as computer science (aimed at physics students), relativity, numerical relativity, and mathematics (this information is also in my CV). My activities included the mentoring of students inside and outside the classroom, the preparation of some lessons, mark assignments and tests, manage course grades, etc. 

As you can see, I am interested in research and academy. I love to teach and help students as I believe that teaching others is one of the best ways to learn. I am also fascinated with all the research process, since the very early stages devoted to reading associated literature, to the final delivery of the job and the little (or big) contribution to humanity's knowledge. 

I had some personal issues at the end of my career, issues that made me look for a better payment than what I was getting as a teacher assistant, so I started to work as a Data Scientist. Throughout my time in the industry, I have shown my employers that I can be as good or sometimes better than others who have more knowledge in this area than I do. I have studied and prepared a lot, I learned how to deliver production-level code in C ++ and Python on my own, I know about machine learning (I am currently creating a model to predict and prevent customer defection), containers and orchestration, and I have gain lots of experience with cloud services, in particular GCP, using tools like BigQuery, Airflow, Data Fusion, etc.

What I am trying to state here is that, I do not want to pursue a Master's Degree to get a job, I have one and I am really interested in what I'm doing and I'm learning a lot. But I do know I could make and be more, and I strongly believe Durham University can help me achieve my goals, as I see it offers a really interesting variety of courses, and the possibility to use some of the Computer Science knowledge I have gathered to study problems in a field such as Astrophysics is most exciting. That being said, I am familiar with Astrophysics and Cosmology, although I have not taken any formal course on those topics besides the one called "Selected topics of Relativity, Cosmology and Gravitation I" (which you can see in my transcript, but this course was more focused on Numerical Relativity than anything else), so I'm currently reading the recommended book to prepare for the specialization: ``An Introduction to modern cosmology'' by Andrew Liddle.

Regarding my reference letters, as you can see I asked for references both from my current and former employers, as well as from Dr Miguel Alcubierre. I know that it is preferred to have two academic references but, to be honest, I never really worked with anyone besides Dr Alcubierre, and I also believe I have become a better scientist and an overall better worker and student because of my time in the industry. Compared to how I was back a few years ago, I now know what I want and I am committed to achieving it no matter what.

I hope I have been able to transmit my interest and my excitement and I urge you to really consider accepting me. Finally, I thank you for taking the time to read my statement and look forward to hearing about your decision. Be sure that, if I enter this program, I will do my best to succeed and learn all I can and honour this great university.

\vspace{4cm}
\hfill \textbf{Uzmar Gómez}

\hfill BSc Physics (UNAM), 

\hfill Data Scientist I at Rackspace Technology
\end{document}