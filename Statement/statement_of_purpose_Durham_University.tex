\documentclass{article}
\usepackage[letterpaper,margin=1in]{geometry}
\usepackage[dvipsnames]{xcolor}
\usepackage{fancyhdr}
\usepackage{tgschola} % or any other font package you like
\usepackage{titlesec}
\usepackage{comment}
\usepackage{amsmath}
\usepackage{setspace}
\usepackage[square,sort,comma,numbers]{natbib}
\usepackage{hyperref}
\hypersetup{
    colorlinks=true,
    linkcolor=blue,
    filecolor=magenta,      
    urlcolor=NavyBlue,
}


\pagestyle{fancy}
\fancyhf{}
\fancyhead[C]{%
  \footnotesize\sffamily
  \yourname:\quad
  {\youremail}}

\newcommand{\soptitle}{Statement of Purpose}
\newcommand{\yourname}{Uzmar G\'omez}
\newcommand{\youremail}{uzmar.gomez@ciencias.unam.mx}
\newcommand{\yourweb}{https://uzmargomez.github.io/}

\newcommand{\statement}[1]{\par\medskip
  \underline{{\textbf{#1:}}}\space
}

\titleformat{\section}{\large\itshape}{\thesection}{1em}{}

\renewcommand{\bibsection}{\section*{Bibliography}}

\begin{document}
\spacing{1.5}
\begin{center}\LARGE\soptitle\\
\large of \yourname\ (Durham University, MSc in Scientific Computing and Data Analysis)
\end{center}

\hrule
\vspace{1pt}
\hrule height 1pt

\bigskip

Here I expose the reasons I have to get into the MSc in Scientific Computing and Data Analysis.

As you can see in my curriculum,  I have a BSc in Physics. In order to get it, I investigated about finite difference methods to solve advective-kind equations, in particular, I needed to evolve an advective-kind equation called the Relativistic Vlasov Equation, on a background Schwarzschild metric. I made a \href{http://oreon.dgbiblio.unam.mx/F/YF3LAD8UIUFR2XRPDE3H6JM2MI44M8RY8QX1RH7K429163KKEV-19694?func=find-b&request=uzmar&find_code=WRD&adjacent=N&local_base=TES01&x=65&y=13&filter_code_2=WYR&filter_request_2=&filter_code_3=WYR&filter_request_3=}{thesis} about this work, under \href{https://sigi.nucleares.unam.mx/sgiicn/people/user/view/id/8}{Dr. Alcubierre}'s guidance, at the \href{https://www.unam.mx/}{National Autonomous University of Mexico (UNAM)}. 

I've been a teacher assistant of subjects such as computer science (aimed at physics students), relativity, numerical relativity, and mathematics (this information is also in my CV). My activities included the mentoring of students inside and outside the classroom, preparation of some lessons, mark assignments and tests, manage course grades, etc. 

As you can see, I am interested in research and academy. I love to teach and help students as I believe that teaching others is one of the best ways to learn. I am also facinated with all the research process, since the very early stages devoted to reading associated literature, to the final delivery of your job and the little (or big) contribution to humanity's knowledge. 

So that may make someone wonder why I go out of the academy. The reason is, I had some personal issues and I needed to get a better payment than what I was perceiving as a teacher assistant, so I started to work as a Data Scientist. I know that perhaps this is not much, but I have shown my employers that I can be as good or even better than others who have more knowledge in this area than I do. I have studied and prepared a lot, managing to learn how to deliver production-level code in C ++ and Python on my own, I have good understanding of machine learning algorithms (I am currently creating a model to predict and prevent a customer from leaving Rackspace), containers and orchestration, and I have started to gain more experience with cloud services, in particular GCP, using tools like BigQuery, Airflow, Data Fusion, etc. As you can see, I am committed to learning as much as I can.

To be clear, I do not want to pursue a Master's Degree to get a job, I currently have one and I am really interested in what I'm doing and I'm learning a lot, but I want to learn more, and I strongly believe Durham University offers a really interesting variety of courses, and the possibility to use some of the Computer Science knowledge I have to study problems in a field such as Astrophysics really exites me. That being said, I am familiar with Astrophysics and Cosmology, although I have not taken any formal course on those topics, so I'm currently reading the recommended book to prepare for the specialization ``An Introduction to modern cosmology'' by Andrew Liddle.

am most interested in the modules on the "Prediction and Learning with Data" theme (maybe because it is the one I'm most familiar with), but having the opportunity to attend courses on the "Artificial Intelligence and Cognitive Science", "Mathematics and Statistics" or "Software Engineering" sets, is something that really excites me about this program.

To be completely honest, I am debating if I would prefer to enter the industry again or pursue a PhD after the successful completion of the MSc. I have always been attracted to the academic environment, doing research and study different and new ideas, while sharing these ideas with students and learning from both teachers and fellow students and teammates, while working alongside top researches on the field. On the other side, I am sure that having an MSc Degree from UCD will give me the possibility to work on a top technology company in Dublin, which is a great opportunity. To be able to pursue either one of those paths is simply very attractive to me.

I hope I have been able to transmit my interest and my excitement, and I look forward to hearing about the decision regarding my acceptance. Please be sure that, if you accept me, I will do my best to succeed and learn all I can from this program and honour this great university.

\end{document}